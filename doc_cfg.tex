% include packages
\usepackage[utf8]{inputenc}
\usepackage[T1]{fontenc}
\usepackage[onehalfspacing]{setspace}
\usepackage{pdfpages}
\usepackage{pgfplots}
\usepackage[english]{babel}
\usepackage[english]{fancyref}
\usepackage{enumerate}
\usepackage{enumitem}				% to easily customize enumeration/itemize environments
\usepackage{listings}				% for code paragraphs (lst)
\usepackage[pict2e]{struktex}		% wg. der Struktogramm-Umgebung, pict2e für beliebige Steigungen von Geraden
\usepackage{graphics}
\usepackage{subcaption}
\usepackage{amsfonts}
\usepackage{amsmath}
\usepackage{amssymb}
\usepackage{dsfont}
\usepackage{tikz}					% to create 2D drawings
\usepackage{tikz-3dplot}			% to create 3D drawings
\usepackage{libertinus}				% font, you can choose lmodern and others
\usepackage{titleps}
\usepackage{fancyhdr}
\usepackage{listings}
\usepackage{color}
\usepackage{colortbl}
\usepackage{tabularx}
\usepackage[open, openlevel=2]{bookmark}
\usepackage{hyperref}
\usepackage[all]{hypcap}
\usepackage{mathtools}
\usepackage{slashbox}
\usepackage{changepage}
\usepackage{lipsum}
\usetikzlibrary{matrix}

% set specific settings
\setlength{\parindent}{0in}
\setcounter{secnumdepth}{5} 		% damit werden Abschnitte bis zur Tiefe 5 mit Nummern versehen (also bis \subparagraph), der Standardwert ist 3
\setcounter{tocdepth}{5}			% damit werden mehr Unterabschnitte ins Inhaltsverzeichnis aufgenommen
% Zählervariable
\newcounter{myenumi}				% nimmt aktuellen Wert von enumi auf

% define colors
\definecolor{gray}{rgb}{0.5,0.5,0.5}
\definecolor{lightgray}{rgb}{0.83, 0.83, 0.83}
\definecolor{dkgreen}{rgb}{0,0.6,0}
\definecolor{deepgreen}{rgb}{0,0.5,0}
\definecolor{limegreen}{rgb}{0.2, 0.8, 0.2}
\definecolor{deepblue}{rgb}{0,0,0.8}
\definecolor{darkblue}{rgb}{0.0,0.0,0.3}
\definecolor{deepred}{rgb}{0.9,0,0}
\definecolor{darkorange}{rgb}{1.0, 0.55, 0.0}
\definecolor{lava}{rgb}{0.81, 0.06, 0.13}

\definecolor{main-color}{rgb}{0.6627, 0.7176, 0.7764}
\definecolor{back-color}{rgb}{0.1686, 0.1686, 0.1686}
\definecolor{string-color}{rgb}{0.3333, 0.5254, 0.345}
\definecolor{key-color}{rgb}{0.8, 0.47, 0.196}

% misc (new commands, lstlisting styles, etc.)
\hypersetup{linkbordercolor=lava}
\hypersetup{citebordercolor=lava}

\lstset
{
	language=Python,
%	frame=tb,
	frame=single,
	tabsize=4,
	breaklines=true,
	aboveskip=3mm,
	belowskip=3mm,
	showstringspaces=false,
	columns=fixed,
	basicstyle={\ttfamily},
	numbers=left,
	numberstyle=\tiny\ttfamily\color{gray},
	breakatwhitespace=true,
	morecomment=[f]{\#},
	stringstyle=\color{deepgreen},
	commentstyle=\color{gray},
	keywordstyle = {\color{deepblue}},
	keywordstyle = [2]{\color{deepblue}},
	keywordstyle = [3]{\color{deepgreen}},
	otherkeywords = {+,-,*,/,//,+=,-=,*=,/=,<,>,==,!=,<=,>=},
	morekeywords = [3]{+,-,*,/,//,+=,-=,*=,/=,<,>,==,!=,<=,>=},
	emph={return},
	emphstyle=\color{lava},
}

\lstdefinestyle{dark}
{
	basicstyle = {\ttfamily \color{main-color}},
	backgroundcolor = {\color{back-color}},
	stringstyle = {\color{string-color}},
	commentstyle=\color{gray},
	keywordstyle = {\color{key-color}},
	keywordstyle = [2]{\color{key-color}},
	keywordstyle = [3]{\color{dkgreen}},
	otherkeywords = {+,-,*,/,//,+=,-=,*=,/=,<,>,==,!=,<=,>=},
	morekeywords = [3]{+,-,*,/,//,+=,-=,*=,/=,<,>,==,!=,<=,>=},
	emph={return},
	emphstyle=\color{lava},
}

\renewcommand{\lstlistingname}{Python Code}
\renewcommand{\lstlistlistingname}{\lstlistingname}
\renewcommand{\labelitemi}{\scriptsize$\blacksquare$}

%\newcommand{\bib}
%{
%	\bibliographystyle{plain}
%	\bibliography{cites}
%	\nocite{*}
%}

\newpagestyle{fancier-toc}
{
	\sethead
	{}{}{}
	\setheadrule{.4pt}
	
	\setfoot
	{}{\thepage}{}
}

\newpagestyle{fancier}
{
	\sethead
	{\toptitlemarks \textit{\thesection} \quad \textit{\sectiontitle}}
	{}
	{\bottitlemarks \textit{\thesection} \quad \textit{\sectiontitle}}
	\setheadrule{.4pt}
	
	\setfoot
	{}{\thepage}{}
}

\DeclarePairedDelimiter\ceil{\lceil}{\rceil}
\DeclarePairedDelimiter\floor{\lfloor}{\rfloor}
\DeclareSymbolFont{italics}{\encodingdefault}{\rmdefault}{m}{it}
\DeclareSymbolFontAlphabet{\mathit}{italics}

\makeatletter
\count@=`a \advance\count@\m@ne
\@whilenum{\count@<`z}\do{%
	\advance\count@\@ne
	\begingroup\lccode`A=\count@
	\lowercase{\endgroup
		\DeclareMathSymbol{A}{\mathalpha}{italics}{`A}%
	}%
}
\count@=`A \advance\count@\m@ne
\@whilenum{\count@<`Z}\do{%
	\advance\count@\@ne
	\begingroup\lccode`A=\count@
	\lowercase{\endgroup
		\DeclareMathSymbol{A}{\mathalpha}{italics}{`A}%
	}%
}
\makeatother

\newcommand{\mybar}[2][3]{{}\mkern#1mu\overline{\mkern-#1mu#2}}
\newcommand{\tild}[2][3]{{}\mkern#1mu\tilde{\mkern-#1mu#2}}
\renewcommand{\bar}{\mybar}
\renewcommand{\labelenumii}{\theenumii}
\renewcommand{\theenumii}{\theenumi.\arabic{enumii}.}